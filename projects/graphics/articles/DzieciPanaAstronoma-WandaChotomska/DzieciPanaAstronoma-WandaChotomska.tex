\documentclass[11pt,a4pape,leqno,twoside]{book}

\usepackage{polski}
\usepackage[utf8]{inputenc}
\title{Dzieci Pana Astronoma}
\author{Wanda Chotomska}
\date{\today}

\begin{document}
\maketitle

% rozdział 1
\chapter{Dzieci Pana Astronoma}
Przy ulicy Astronomów,\\
w jednym z bardzo wielu domów,\\
mieszkał sobie razem z żoną\\
roztargniony pan Astronom.\\
W domu nie widziało go się,\\
bo wciąż błądził po Kosmosie.\\
Przez ogromne teleskopy,\\
których zresztą miał na kopy,\\
przez soczewki i lunety\\
badał gwiazdy i planety.\\ \vspace{0.1cm}

Kiedyś, kiedy przez teleskop\\
patrzył właśnie w dal niebieską,\\
kiedy w nos mu Księżyc świecił,\\
głos usłyszał: - Mamy dzieci!\\
Urodziła nam się naraz\\
bardzo miłych bliźniąt para!\\
Syn i córka - on i ona!\\
Jakie nadać im imiona?\\
Drogi mężu, pomyśl krzynę,\\
jak się ma nazywać synek?\\
- Teleskopek, droga żono -\\
rzekł natychmiast pan Astronom.\\
- A córeczka? - Teleskopka!\\
Teleskopki, no i kropka!\\ \vspace{0.1cm}

Mija roczek, drugi, trzeci -\\
jak na drożdżach rosną dzieci.\\
Rosną we dnie, rosną nocą,\\
jedzą, piją oraz psocą.

%rozdział 2
\chapter{Dzieci puszczają zajączki}
Przy ulicy Astronomów\\
słońce puka w okna domów.\\
Dzień się zbudził, zapiał kogut,\\
kot przeciąga się na progu,\\
kuy gdaczą, pies zaszczekał,\\
mleczarz przywiózł wózek mleka,\\
więc bliźnięta wstały z łóżek\\
i wychodzą na podwórze.\\ \vspace{0.1cm}

Teleskopka karmi kotka,\\
siedząc na kamiennych schodkach.\\
Teleskopek na nią zerka:\\
- Przynieś szubko dwa lusterka!\\
Jedno lustro wziął do rączki\\
i już puszcza nim zajączki.\\
Teleskopka, wielka śmieszka,\\
świeci drugim w okna mieszkań,\\
a tu okna ktoś otworzył\\
i wołają lokatorzy:\\
- Dzieci pana Astronoma\\
to Gomora i Sodoma\\ \vspace{0.1cm}

Zezłościły się sąsiadki:\\
- te bliźnięta to gagatki!\\
Co za dzieci! Co za dzieci!\\
Zaraz macie przestać świecić!\\
Astronoma obudzono.\\
Wyszedł z domu pan Astronom,\\
kiwnął głową, kiwnął bródką\\
i odezwał się cichutko:\\
- By zakończyć ten ambaras,\\
na przechadzkę chodźcie zaraz.\\
Ustawiajcie się w dwuszereg\\
i idziemy na spacerek.

%rozdział 3
\chapter{Dzieci śpiewają piosenkę}
Nikt nie wyciera oczu chusteczką,\\
nikomu z oka nie kapie łza,\\
wszyscy od dzisiaj grają w słoneczko,\\
bo to jest świetna gra.\\ \vspace{0.1cm}

Grasz w "słoneczko"?\\
Gram w "słoneczko"!\\
Masz "słoneczko"?\\
Mam "słoneczko"!\\
Jestem słoneczny\\
i uśmiechnięty\\
od czubka głowy\\
do końca piety.\\
Słoneczny uśmiech\\
na twarzy mam\\
i z całym światem\\
w "słoneczko" gram.\\
Masz "słoneczko"?\\
Mam "słoneczko"!\\
Prawda, że to fajna gra?\\ \vspace{0.1cm}

A jak ktoś przegra, a jak ktoś skusi,\\
słoneczny uśmiech słońce mu da,\\
pięknym uśmiechem zapłacić musi,\\
bo to jest taka gra.\\ \vspace{0.1cm}

Szli, śpiewając, przez ulice,\\
podziwiali kamienice,\\
obejrzeli dwieście wystaw,\\
samochodów prawie trzysta,\\
pięć pomników, kominiarza,\\
straż ogniową i malarza.\\ \vspace{0.1cm}

Wreszcie rzekły  Teleskopki:\\
- Tato, nas juz bolą stopki!\\
Straszny upał dziś na mieście,\\
więc na lody chodźmy wreszcie\\ \vspace{0.1cm}

I Astronom wraz z dzieciarnią\\
siadł w ogródku przed kawiarnią,\\
przy okrągłym żółtym stole\\
pod pasiastym parasolem.\\ \vspace{0.1cm}

Jedli lody z poziomkami,\\
pili wodę z bąbelkami.\\
Właśnie przy tej wodzie z gazem -\\
zapytali naraz razem:\\
- Tato, czemu słońce świeci?\\
Tata zerknął na swe dzieci\\
i łyknawszy łyczek kawy,\\
wykład zrobił im ciekawy.

%rozdział 4
\chapter{Pan Astronom mówi o słońcu}
Dlaczego Słońce świeci?\\
To jasne jest od razu,\\
jeśli się wie, że Słońce\\
to wielka kula gazów.\\ \vspace{0.1cm}

Te gazy wciąż się tłoczą,\\
te gazy wciąż są w ruchu\\
i zarzą się, i prażą\\
w słonecznym wielkim brzuchu.\\ \vspace{0.1cm}

Słońce to kula gazów\\
bez przerwy rozżarzona.\\
Stąd właśnie płynie światło,\\
stąd płynie ciepło do nas.\\ \vspace{0.1cm}

Popatrzcie naokoło:\\
to wszystko, co widzicie -\\
świat roślin, ludzi, zwierząt -\\
Słońcu zawdzięcza życie.\\ \vspace{0.1cm}

Skończył tato swe wywody,\\
dzieci zaś skończyły lody,\\
zapłacili i po chwili\\
od stolika się ruszyli.\\ \vspace{0.1cm}

Po godzinie byli w domu.\\
Przy ulicy Astronomów.\\ \vspace{0.1cm}

Teleskopka drzwi otwiera.\\
- Co będziemy robić teraz?\\
Teleskopek krzyknął pierwszy:\\
- Wiesz co? Napiszemy wierszyk!\\
Wymyślimy różne rymy -\\
wiersz o Słońcu ułozymy!\\ \vspace{0.1cm}

Podskoczyła Teleskopka:\\
- To dopiero będzie szopka!\\
Słowo daję, jak tu stoję,\\
napiszemy wiersz oboje\\
i ten wiersz wyślemy w liście.\\
- Gdzie? - Do Słońca oczywście

%rozdział 5
\chapter{Dzieci piszą list do Słońca}
\textit{"Kochane Słoneczko,}\\
\textit{co świecisz na niebie -}\\
\textit{list czerwoną kredką}\\
\textit{piszemy do Ciebie.}\\ \space{0.1cm}

\textit{Słoneczko-lampeczko,}\\
\textit{kochamy Cię za to,}\\
\textit{że nas bardzo ładnie}\\
\textit{opalasz przez lato.}\\ \vspace{0.1cm}

\textit{Słoneczko-piecyku,}\\
\textit{bardzo Cię prosimy:}\\
\textit{ogrzewaj nas mocniej}\\
\textit{również w czasie zimy.}\\ \vspace{0.1cm}

\textit{Nie żałuj nam ciepła,}\\
\textit{nie żałuj promieni}\\
\textit{ani w dni wiosenne,}\\
\textit{ani w jesieni.}\\ \vspace{0.1cm}

\textit{Grzej, ile masz siły,}\\
\textit{tak jak tylko możesz,}\\
\textit{żeby przez rok cały}\\
\textit{upał był na dworze.}\\ \vspace{0.1cm}

\textit{Niech będzie gorąco,}\\
\textit{żeby dla ochłody}\\
\textit{musieli bez przerwy}\\
\textit{kupować nam lody!}\\ \vspace{0.1cm}

\textit{Więcej już nie mamy}\\
\textit{żądań ani życzeń,}\\
\textit{więc Cię całujemy}\\
\textit{w każdy Twój promyczek".}\\ \vspace{0.4cm}

Teleskopek z Teleskopką\\
list skończyli piękną kropką\\
i oboje zamaszyście\\
podpisali się na liście.\\ \vspace{0.1cm}

Odetchnęli. Ale zaraz\\
nowy zaczął się ambaras.\\
Bo jak wysłać list do Słońca?\\
Kogo użyć w roli gońca?\\
Listonosza?\\
- Nie!\\
Stwierdzono,\\
że tu na nic pan listonosz.\\ \vspace{0.1cm}

Teleskopek kasę liczy.\\
- Może nadać list lotniczy?\\ \vspace{0.1cm}

Hm... niestety, też nie da się,\\
bo za mało groszy w kasie...\\ \vspace{0.1 cm}

Uradzono koniec końcem:\\
- Niech balonik będzie gońcem!\\ \vspace{0.1cm}

I poleciał list balonem\\
nad podwórzem, nad balkonem,\\ \vspace{0.1cm}

nad kominiem - hen, wysoko,\\
gdzie nie dojrzy ludzkie oko.\\ \vspace{0.1cm}

Teleskopek przez teleskop\\
patrzy za nim w dal niebieską.\\
- Co tam widzisz?\\
- Mamy widok,\\
tata z mamą tutaj idą!\\ \vspace{0.1cm}

Mama tylko kiwa głową,\\
tata wzdycha: - Daję słowo,\\
słowo daję, moi słoci,\\
żadna z planet tak nie psoci!\\
A bliźnięta krzyczą na to:\\
- Dziwne rzeczy mówisz, tato!\\
Żadna z planet?\\
- Co to: "planet"?\\
To jest słowo nam nie znane.\\
Niech nam tata wytłumaczy,\\
co to dziwne słowo znaczy!

%rozdział 6
\chapter{Pan Astronom mówi o planetach}
Wkoło Słońca biega sobie\\
dziewięć planet dużych -\\
w dzień i w nocy wciąż się kręcą,\\
stale są w podróży\\ \vspace{0.1cm}

Krążą Ziemia, Mars i Jowisz,\\
Merkury z Plutonem,\\
Wenus, Saturn, Uran, Neptun -\\
wszystkie w jedną stronę.\\ \vspace{0.1cm}

Dziewięc planet obok siebie\\
kręcą się bez końca\\
i choć każda z nich jest inna,\\
krążą wokół Słońca.\\  \vspace{0.1cm}

Jak wiadomo, od stuleci\\
dużo zabaw znajdą dzieci.\\
Lubią bawić się w "komórki",\\
pędzą "z górki na pazurki",\\
znają "klasy", "ślepą babkę",\\
"fanty", "kółko", "boćka-żabkę",\\dobrze wiedzą, co to "berek"\\
oraz innych zabaw szereg.\\ \vspace{0.1cm}

Naszych miłych bliźniąt para\\
też zna zabaw co niemiara,\\
lecz na wszystko kręcą nosem\\
i znudzonym mówią głosem:\\
- Nie zdajecie sobie sprawy,\\
jak nam zbrzydły te zabawy!\\
Wymyślimy coś nowego!\\
Coś zupełnie nieznanego!\\
Zbiegła się kolegów chmara\\
i pytają wszyscy naraz:\\
- Co to będzie za zabawa?\\
- Czy wesoła? Czy ciekawa?\\
Teleskopek odparł z mety:\\
- Zabawimy się w planety!!!\\
Słońcem będzie Teleskopka.\\
- Teleskopka? A to szopka!

%rozdział 7
\chapter{Dzieci bawią się w planety}
Wszyscy wpadli w wielki zapał.\\
Każdy prędko patyk złapał,\\
porobili różne koła\\
i gra toczy się wesoła.\\
Bez przecinka i bez kropki\\
pędzą wokół Teleskopki,\\
ona krzyczy zaś w zapale:\\
- Jestem Słońce! Świecę stale!\\
Żarem bucham z głębi brzucha!\\
Każdy musi mnie się słuchać!\\
Jestem królem wszystkich planet!\\
Nigdy świecić nie przestanę!\\ \vspace{0.1cm}

Macie robić, co rozkażę,\\
więc wam każę paść na twarze!!!\\
JOWISZ wrzasnąl z groźną miną:\\
- Opamiętaj się, dziewczyno!\\
Zastanowił się MERKURY:\\
- Przecież ona plecie bzdury!\\
MARS, strasznego robiąc marsa,\\
krzyknął gromko:\\
- To ci farsa!\\
WENUS się puknęła w ciemię,\\
SATURN łokciem stuknął ZIEMIĘ,\\
ZIEMIA z biegu wytrącona\\
uderzyła w nos PLUTONA,\\
MARS chciał skoczyć na ratunek,\\
ale zderzył się z NEPTUNEM,\\
i po chwili, moi mili,\\
wszyscy strasznie się czubili.\\
Tak się bili, że na końcu\\
JOWISZ guza nabił SŁOŃCU.\\
SŁOŃCE się zalało łzami:\\
- Ja się nie chcę bawić z wami!

%rozdział 8
\chapter{Dzieci stawiają pomnik}
- Przy ulicy Astronomów\\
stoi bardzo dużo domów,\\
lecz brakuje tu pomnika\\
Mikołaja Kopernika -\\
tak powiedział pan Astronom,\\
kiedy szedł na spacer z żoną.\\ \vspace{0.1cm}

Popatrzyła bliźniąt para:\\
- Nie ma, ale będzie zaraz,\\
będzie pomnik Kopernika,\\
bo Kopernik wart pomnika!\\ \vspace{0.1cm}

Skrzyknął chłopców Teleskopek:\\
- Do mnie, chłopcy! Tu, galopem!\\
Co Kopernik robił? Wiecie?\\
- On coś odkrył pierwszy w świecie...\\
- Mnie obiło się o uszy,\\
że Kopernik Ziemię ruszył.\\
Wstrzymał Słońce, ruszył Ziemię,\\
polskie wydało go plemię.\\
- Wstrzymał Słońce? Co to znaczy?\\
- Ja wam mogę wytłumaczyć! -\\
tak oznajmił Teleskopek,\\
więc postawmy tu dwukropek:\\
- Tata mówił, że przed laty\\
ludzie się nie znali na tym\\
i nie wiedział nikt z uczonych,\\
jak ten świat jest urządzony.\\
"Słońce krąży wokół Ziemi" -\\
powtarzali ci uczeni\\
i pojęcia żaden nie miał,\\
że to właśnie krąży Ziemia.\\ \vspace{0.1cm}

Krąży, krąży i bez końca\\
kręci się dokoła Słońca!\\ \vspace{0.1cm}

To Kopernik odkrył pierwszy\\
i stąd właśnie jest ten wierszyk:\\
"Wstrzymał Słońce, ruszył Ziemię -\\
polskie wydało go plemię".\\
Zróbmy pomnik Kopernika,\\
bo Kopernik wart pomnika!\\ \vspace{0.1cm}

Pomyśleli chłopcy krzynkę,\\
postawili chłopcy skrzynkę.\\
- To jest cokół.\\
- A gdzie posąg?\\
- W prześcieradle już go niosą.\\
Od fryzury aż po pięty\\
w prześcieradło owinięty.\\
- Ty, a po co prześcieradło?\\
- Jak to po co? Żeby spadło.\\
Żeby spadło w tym momencie,\\
gdy nastąpi odsłonięcie.\\
- Teleskopko, chodź tu do nas!\\
Odsłoń pomnik astronoma!\\
Teleskopko, odsłoń pomnik\\
i o mowie nie zapomnij!\\ \vspace{0.1cm}

Teleskopka kwiaty taszczy.\\
- Przypadł mi w udziale zaszczyt,\\
zaszczyt przypadł mi w udziale...\\
Dobrze mówię?\\
- Doskonale!\\
Pociągnęła prześcieradło,\\
prześcieradło raz-dwa spadło...\\
Patrzą z okien lokatorzy:\\
- Co to? Kto to? Pomnik ożył!\\
Teleskopek na cokole?\\
Co się dzieje tam na dole?\\
To jest jakaś nowa szopka!\\
- Nie - odrzekła Teleskopka -\\
to Kopernik w młodych latach,\\
gdy był w wieku mego brata...\\ \vspace{0.1cm}

Przy ulicy Astronomów\\
zasypiają okna domów,\\
pod kołdrami z srebrnej blachy\\
zasypiają wszystkie dachy.\\
Wiatr obłoki do snu zgarnia,\\
śpią kominy i latarnia,\\
sennie mruczą drzewa śpiące,\\
koty kończą koci koncert\\
i zagląda kotom w oczy\\
Księżyc, co po niebie kroczy.\\ \vspace{0.1cm}

W noc Księżycem wysrebrzoną\\
patrzył właśnie pan Astronom,\\
a tu z domu zakamarków\\
wyszła para nocnych\\
marków.\\
Teleskopek z Teleskopką\\
zatupali bosą stopką\\
i oboje jak nie wrzasną:\\
- Księżyc nam nie daje\\
zasnąć!\\
Księżyc łazi nam po głowie -\\
o Księżycu nam opowiedz!

%rozdział 9
\chapter{Pan Astronom mówi o Księżycu}
- Dookoła Ziemi\\
ciągle sobie biega\\
Księżyc - naszej Ziemi\\
najbliższy kolega.\\ \vspace{0.1cm}

Cały miesiąc musi\\
biegać dookoła,\\
zanim raz okrążyć\\
naszą Ziemię zdoła.\\ \vspace{0.1cm}

Słońce go oświetla -\\
tam gdzie Słońce zerka,\\
światło się odbija\\
tak jak od lusterka.\\ \vspace{0.1cm}

Sam Księżyc nie świeci,\\
to, co widać z Ziemi,\\
jest właśnie odbiciem\\
słonecznych promieni.\\ \vspace{0.1cm}

*\\ \vspace{0.1cm}

Przerwał wykład pan Astronom,\\
siedzi z miną zamyśloną.\\
Podskoczyła bliźniąt para,\\
obydwoje piszczą naraz:\\
- Tato, zdradź nam tajemnicę,\\
czy gdzieś jeszcze są księżyce?\\ \vspace{0.1cm}

- Dużo jest księżyców w świecie -\\
większość planet ma je przecież.\\
Dwa księżyce Mars posiada,\\
Neptun także dwoma włada,\\
pięć okrąża Uran wokół,\\
Saturn - dziesięć ma przy boku.\\
Teraz, moi drodzy, cisza,\\
dochodzimy do Jowisza.\\
Tylko sobie wyobraźcie -\\
on księżyców ma dwanaście!\\
Teleskopka z palcem w buzi\\
mruczy sennie: - Cały tuzin!\\
Teleskopek trze oczęta...\\
Bądźcie cicho... Śpią bliźnięta...

%rozdział 10
\chapter{Dzieciom śnią się księżyce}
Śni się ulica i śnią się dachy,\\
dachy, ulice, ulice,\\
a nad dachami ze srebrnej blachy\\
srebrne i złote księżyce.\\ \vspace{0.1cm}

- O - miauczą koty. - O - piszczą koty. -\\
Czy nam się śni, czy nie śni?\\
Tu księżyc srebrny, tam księżyc złoty\\
drogę na niebie kreśli.\\ \vspace{0.1cm}

Psy wyszły z budy. - O, co to znaczy -\\
okropnie kręcą szyją.\\
Tuzin księżyców każdy zobaczył,\\
więc do księżyców wyją.\\ \vspace{0.1cm}

- O - kocie oczy. - O - kocie oczy,\\
a w kocich oczach zieleń. -\\
Tuzin księżyców na nas się toczy.\\
O - tego już za wiele!\\ \vspace{0.1cm}

Popatrz na koty i nic się nie dziw,\\
że w górę zadarły noski -\\
przecież na każdym księżycu siedzi\\
wąsaty imć pan Twardowski.\\ \vspace{0.1cm}

Psy oniemiały, proszę rodziców,\\
koty dostały zeza,\\
wszyscy Twardowscy spadli z księżyców\\
i tańczą poloneza!\\ \vspace{0.1cm}

Przy ulicy Astronomów\\
zamieszanie w całym domu -\\
trzy kuzynki i dwie ciotki\\
pieką torty i szarlotki,\\
pianę biją dwaj wujkowie,\\
dziadek mak trze na makowiec,\\
stryjek kręci jajka z cukrem,\\
babcia robi babkę z lukrem.\\
Pan Astronom zamyślony\\
szepce w kącie coś do żony.\\
Tajemnicze mają miny.\\
- Jutro dzieci urodziny!\\
Co w prezencie na to święto,\\
droga żono, dać bliźniętom?\\
Rzekła żona: - moim zdaniem,\\
trzeba kupić im ubranie.\\
- Nie! Nie! - krzyknął pan Astronom. -\\
To zbyt zwykłe, moja żono,\\
prędko wymyśl coś innego,\\
coś takiego... niezwykłego!\\
- Może rower? Dwa rowery?\\
- Nie! Nie! Jesli mam być szczery,\\
to się zawsze boję szczerze,\\
gdy ktoś jedzie na rowerze.\\
Tu się wtrącił wuj ze stryjkiem:\\
- Kupmy dzieciom loteryjkę.\\
- Nie! - krzyknęły obie ciotki. -\\
Lepiej łyzwy albo wrotki!\\
Babcia pokręciła głową:\\
- O, nie, wrotki to niezdrowo!\\
Lepiej pułk żołnierzy z blachy.\\
Dziadek wtrącił: - Lepiej szachy. -\\
Lecz ogólny słysząc sprzeciw,\\
orzekł: - Zapytajmy dzieci!\\
Para bliźniąt po namyśle\\
rzekła chytrze: - Mówiąc ściśle,\\
nam niczego nie potrzeba\\
prócz drobnostki - gwiazdki z nieba.\\
Proszę wujków, cioć i taty -\\
chcemy gwiazdkę - nic poza tym!\\ \vspace{0.1cm}

Puknął się astronom w czoło:\\
- Już wiem, co wam kupię! ZOO!\\
- Lecz co ZOO ma do gwiazdki???\\
- Co? Słuchajcie opowiastki...

%rozdział 11
\chapter{Pan Astronom mówi o gwiazdach}
- Nasze Słońce ma rodzinę,\\
najliczniejszą z rodzin -\\
każda gwiazda, ile jest ich,\\
w skład rodziny wchodzi.\\ \vspace{0.1cm}
Słońce jest podobne gwiazdom -\\
każda z gwiazd z osobna,\\
od największej do najmniejszej,\\
Słońcu jest podobna.\\ \vspace{0.1cm}

Oczywiście wiem, że zaraz\\
ktoś pytanie da mi:\\
"Czyżby Słońce było gwiazdą,\\
a gwiazdy słońcami?"\\ \vspace{0.1cm}

Tak - odpowiem. - Tak jest właśnie,\\
macie rację, dzieci,\\
i dlatego każda gwiazda\\
tak jak Słońce świeci.\\ \vspace{0.1cm}

Tutaj nastąpiła cisza,\\
bo Astronom się zadyszał.\\
Niecierpliwią się bliźnięta:\\
- Tato, a gdzie te zwierzęta?\\
Wspomniałeś coś o ZOO,\\
a o gwiazdach mówisz w koło!\\
Pan Astronom kiwnął bródką:\\
- Będzie ZOO! - odparł krótko. -\\
Zaraz przyjdzie na nie pora,\\
ZOO będzie w gwiazdozbiorach.

%rozdział 12
\chapter{Pan Astronom mówi o gwiazdozbiorach}
- Gdy na niebo patrzą ludzie\\
z wsi, miasteczek oraz miast,\\
widzą różne gwiazdozbiory -\\
gwiazdozbiory - zbiory gwiazd.\\ \vspace{0.1cm}

Bardzo dużo gwiazdozbiorów,\\
moi drodzy, dobrze znam,\\
lecz opowiem tylko o tych,\\
które są najbliższe nam.\\ \vspace{0.1cm}

Słońce, Księżyc i planety -\\
cały nasz najbliższy świat -\\
wśród tuzina gwiazdozbiorów\\
krąży od miliardów lat.\\ \vspace{0.1cm}

Te dwanaście gwiazdozbiorów,\\
co pas tworzą wokół nas,\\
to jest właśnie ów zwierzyniec,\\
czyli ZODIAK - proszę was!\\ \vspace{0.1cm}

*\\ \vspace{0.1cm}

Zaklaskały dzieci w ręce:\\
- Tato, powiedz nam coś więcej!\\
To historia wprost niezwykła!\\
O zwierzyńcu zrób nam wykład!

%rozdział 13
\chapter{Pan Astronom mówi o zodiaku}
- W tym niebieskim zwierzyńcu,\\
proszę zacnych słuchaczy -\\
cuda można oglądać,\\
cuda można zobaczyć!\\ \vspace{0.1cm}

Każdy okaz na pokaz,\\
hokus-pokus i cud!\\
Różne cudy na pudy,\\
a tych cudów jest w bród:\\ \vspace{0.1cm}

BARAN stoi na przodzie,\\
choć ma rogi - nie bodzie.\\
Zaraz za nim BYK byczy,\\
który nigdy nie ryczy.\\
Dalej widać BLIŹNIĘTA,\\
RAK im depce po piętach.\\
LEW ogląda się srogo,\\
PANNA niesie mu ogon.\\
WAGA patrzy w jej stronę,\\
wisi tuż przed SKORPIONEM.\\
Jest i STRZELEC w tym tłumie,\\
ale strzelać nie umie.\\
KOZIOROŻEC nie bryka,\\
lecz prowadzi WODNIKA,\\
a ten, patrząc za siebie,\\
liczy RYBY na niebie,\\
przed którymi ów BARAN\\
stoi właśnie jak taran...\\
Tu zamyka się koło\\
oraz wykład o ZOO.\\ \vspace{0.1cm}

Skończył mówić pan AStronom,\\
a bliźniętom oczy płoną.\\
Teleskopka, wniebowzięta,\\
mówi: - W ZOO są Bliźnięta!\\
Para Bliźniąt - tak jak w domu\\
przy ulicy Astronomów!\\
Teleskopek marszczy czoło,\\
myśli o niebieskim ZOO,\\
aż wykrzyknął sam do siebie:\\
- Rak-nieborak jest na niebie!\\
W moim guście taka draka -\\
ujrzeć w niebie - nieboraka!\\
Koziorożec, Lew, Byk, Baran -\\
podskoczyła nasza para:\\
- Chodź nam ZOO kupić, tato.\\
Przecież my czekamy na to!\\
Nie pomogą żadne miny -\\
jutro nasze urodziny.\\
Obiecałeś na to święto\\
ZOO kupić swym bliźniętom.\\ \vspace{0.1cm}

Roztargniony pan Astronom\\
zrobił minę przerażoną,\\
głuchym głosem jęknął: - Oo!\\
Za co ja wam kupię ZOO?\\
Rozpłakały się bliźnięta:\\
- Obiecałeś! Na Jowisza -\\
cały dom to przecież słyszał!\\
Yy... cacanka-obiecanka...\\
Yy... a teraz niespodzianka...\\
My nie chcemy niespodzianek!\\
My jesteśmy oszukane!\\ \vspace{0.1cm}

Pan Astronom słysząc jęki\\
chudy portfel wziął do ręki,\\
mrucząc z miną niewesołą:\\
- Ha! Spróbujmy kupić ZOO!\\
Teleskopka z Teleskopkiem\\
z miejsc ruszyli się galopkiem,\\
jak rakiety mknąc do przodu\\ \vspace{0.1cm}

polecieli do Ogrodu\\
i choć późna była pora,\\
odszukali dyrektora.\\
Pan Astronom kiwnął bródką,\\
a bliźnięta rzekły krótko:\\
- Tatuś ma do pana sprawę.\\
Jaką? Powie sam niebawem.\\
Zostawimy panów samych\\
i zwierzyniec pozwiedzamy.\\ \vspace{0.1cm}

Poszły dzieci zwiedzać ZOO,\\
przyglądały się bawołom,\\
obejrzały panter kilka,\\
stado żubrów, słonia, wilka,\\
nosorożca, sępa, lisa,\\
dwa lamparty i tygrysa,\\
papug nie wiadomo ile,\\
pięć pawianów, trzy goryle,\\
krokodyle, antylopy,\\
dziki, żbiki oraz szopy.\\
Zatrzymały się przed strusiem,\\
przyglądając chwilę mu się,\\
potem okiem niezbyt czułym\\
obrzuciły lwa i muły,\\
zebry, bobry i kojota,\\
który się po klatce miotał.\\
Misia, morsa oraz fokę\\
ledwie zaszczyciły wzrokiem,\\
wykrzywiły się do lamy\\
i ziewnęły:\\ \vspace{0.1cm}

- Dosyć mamy!\\
Odwróciły się na pięcie:\\
- Co po takim nam prezencie?\\
Cóż - zwierzęta jak zwierzęta -\\
ale gdzie tu są Bliźnięta?\\
Koziorożca wśród rogaczy\\
i ze świecą nie zobaczysz,\\
szukaliśmy Strzelca z Wagą -\\
Wagi nie ma i nie ma go,\\
Panny możesz szukać sto dni,\\ \vspace{0.1cm}

w żadnej klatce nie tkwi Wodnik...\\
Te okazy, co są w klatkach,\\
bawić mogły prapradziadka,\\
ale nie nas, proszę taty,\\
myśmy się poznali na tym!\\
Takie ZOO mamy w nosie -\\
lepsze ZOO jest w Kosmosie!\\ \vspace{0.1cm}

- Racja! - krzyknął pan Astronom\\
z miną nagle rozjaśnioną. -\\
Po tym, coście powiedzieli,\\
mogę schować swój portfelik.\\
Ach, jakże mi jest wesoło,\\
że nie chcecie tego ZOO!\\
A bliźnięta krzyczą na to:\\
- Tamto ZOO kup nam, tato!\\
A jak nie chcesz kupić nam, to -\\
pokaż nam choć ZOO tamto!\\
Wiemy, że tam trudny dojazd,\\
więc nam zrób - kosmiczny pojazd!!!\\
- Hm... - powiedział pan Astronom\\
z miną wielce zamyśloną. -\\
Rola ojca nie jest łatwa,\\
gdy chce w Kosmos lecieć dziatwa.\\
Lecz cóż robić - jeśli chcecie,\\
pomyślimy o rakiecie.\\
Pomyślimy o niej w domu\\
przy ulicy Astronomów...\\ \vspace{0.1cm}

*\\ \vspace{0.1cm}

Poszli myśleć. Dotąd myślą,\\
więc prosimy was o ciszę.\\
Co wymyślą - to nam przyślą,\\
a ja znowu to opiszę.

\chapter{Dane książki}
Warszawa\\ \vspace{0.5cm}
ISBN 83-7153-045-5
\end{document}