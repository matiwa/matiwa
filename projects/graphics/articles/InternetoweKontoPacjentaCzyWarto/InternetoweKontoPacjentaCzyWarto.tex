\documentclass[11pt,a4paper]{article}

\usepackage{polski}
\usepackage[utf8]{inputenc}
\title{Czy warto mieć Internetowe Konto Pacjenta?}
\author{Mateusz Iwańczak}
\date{\today}

\begin{document}
\maketitle

\begin{abstract}
\textbf{W niniejszym artykule chce poddać rozważaniom Internetowe Konto Pacjenta, czy jest ono w ogóle dla nas potrzebne.}
\end{abstract}

% pierwsza sekcja
\section{IKP w codzienności}\label{sec:IKP}
Człowiek od zawsze pragnął być nieśmiertelny, chciał mieć dostatnie życie i być zdrowym z każdym tchnieniem. Codzienność poddawała jednak konsekwentnie jego marzenia, dając noworodków, a zabierając żyjących w różnym wieku. Stosunkowo długość życia nie była wielka. Wynikało to z wielu czynników. Należały do tego grona zmagania z przyrodą, niebezpieczne zwierzęta, trujące rośliny, brak higieny i lekarstw, a co za tym idzie właściwego leczenia bolączek cielesnych. Są potencjalnie inne przyczyny tego stanu rzeczy. Warto pochylić się nad aspektem zdrowia. 

% druga sekcja
\section{Zdrowie}\label{sec:zdrowie}
Aspekt zdrowia towarzyszy codzienności każdego człowieka. Sam termin zdrowie jest określany jako stan pełnego, fizycznego, umysłowego i społecznego dobrostanu, a także zadowolenia z niego. Wszyscy pragną być zdrowym jak ryba. To powiedzenie nie jest przypadkowe w odniesieniu do ryb. Ma miejsce niezwykle rzadko, by te oto zwierzęta chorowały. Określając osobę zdrową jak ryba definiowany jest ktoś pełen energii

\end{document}