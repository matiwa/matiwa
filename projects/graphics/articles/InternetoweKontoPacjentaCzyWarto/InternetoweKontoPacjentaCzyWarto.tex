\documentclass[11pt,a4paper]{article}

\usepackage{polski}
\usepackage[utf8]{inputenc}
\title{Czy warto mieć Internetowe Konto Pacjenta?}
\author{Mateusz Iwańczak}
\date{\today}

\begin{document}
\maketitle

\begin{abstract}
\textbf{W niniejszym artykule chce poddać rozważaniom Internetowe Konto Pacjenta, czy jest ono w ogóle dla nas potrzebne.}
\end{abstract}

% pierwsza sekcja
\section{Zdrowie w codzienności}
Człowiek od zawsze pragnął być nieśmiertelny, chciał mieć dostatnie życie i być zdrowym z każdym tchnieniem. Codzienność poddawała jednak konsekwentnie jego marzenia, dając noworodków, a zabierając żyjących w różnym wieku. Stosunkowo długość życia nie była wielka. Wynikało to z wielu czynników. Należały do tego grona zmagania z przyrodą, niebezpieczne zwierzęta, trujące rośliny, brak higieny i lekarstw, a co za tym idzie właściwego leczenia bolączek cielesnych. Są potencjalnie inne przyczyny tego stanu rzeczy. Warto pochylić się nad aspektem zdrowia. 

% druga sekcja
\section{Przysłowia związane ze zdrowiem}
\subsection{Trudne frazy o zdrowiu}
Aspekt zdrowia towarzyszy codzienności każdego człowieka. Każdy język ma swoje skróty, powiedzenia i przysłowia. Są to frazy, które  obcokrajowcom spędzają sen z powiek. Ci, którzy posługują się danym językiem od dziecięcych lat, również mają problemy z odnalezieniem sensu tych zwrotów. Dobrze jest pochylić się na początek nad sensem i znaczeniem, a następnie używać zwrotów w życiu codziennym.

\subsection{Polacy a frazy}
Nie jest obce dla wielu Polaków powiedzenia takie jak: uparty jak osioł, pracowity jak mrówka, wysoki jak żyrafa, czy ... zdrów jak ryba. Owe porównania mają swój sens i nie są przypadkowe. Dane zwierzęta charakteryzują się takimi określeniami.

\subsection{Zdrowie ryb}
Żaden człowiek nie spotkał się z tym, by ryby chorowały - przynajmniej zdarza się to niezwykle rzadko. Sytuacje takie najczęściej mają miejsce, gdy dochodzi do zanieczyszczeń wód. Pojawia się wtedy widok pływających ryb z brzuchem zorientowanym do góry. Osoba, która jest porównywana do zdrowej jak ryba, charakteryzuje się pełnią energii, nie ma choroby i emanuuje zdrowiem.

\subsection{Zdrowa ryba w diecie}
Ryba, okazując zdrowie, jest warta tego, by stała się częścią codziennej diety. Zawiera kwasy tłuszczowe omega i DHA, a także witaminę D. Ta oto witamina jest pożądana dla rozwoju i trwałości kości. Z pewnością warto wspomnieć o minerałach, którymi są wapń, jod, selen, magnez i fluor. Są to składniki potrzebne dla organizmu, lecz człowiek nie da rady ich sam wytworzyć. Musi je dostarczyć zewnętrznie poprzez pokarm lub suplementy diety.

\subsection{Tran i olej z wątroby rekina poza rybami}
Ludzka dieta poza rybim mięsem może być urozmaicona w tran i olej z wątroby rekina. Olej posiada właściwości lecznicze i wzmacnia organizm, uodparnia w czasie jesieni i zimy. Jedzenie ryb poprawia zmysł wzroku, unormowywuje cere, optymalizuje ciśnienie krwi i - co ważne w dzisiejszych czasach - poprawia pamięć i koncentracje.

\subsection{Ryba dla zdrowia}
Do tygodniowego jadłospisu warto jest dodać ryby, gdy towarzyszy miażdżyca, nadciśnienie i choroby układu krążenia. Takie zdanie mają zarówno dietetycy jak i lekarze. Ponadto zalecają, by pojawiały się one w menu nawet trzy razy na jeden tydzień.

\section{Zdrowie w literaturze}
Bez zagłębiania się w dzieła literackie wystarczy przytoczyć jeden przykład, który wyznacza wartość zdrowia. Fraszka "Na zdrowie" Jana Kochanowskiego uwydatnia to, że jest to rzecz najcenniejsza, którą ma człowiek w życiu. Poleca, by o nie dbać i je docenić.

\section{Zdrowie}
\subsection{Znaczenie zdrowia w społeczeństwie}
Wszyscy chcą tryskać zdrowiem, czego niezbitym dowodem jest składanie sobie życzeń urodzinowych, imieninowych, świątecznych, czy noworocznych. Chyba każdy składał drugiemu życzenia, które rozpoczynały się słowem zdrowie. Małżonkowie, składając sobie przysięgę, chcą dochować sobie jej w zdrowiu i w chorobie. Życzenia weselne zawierają też słowo zdrowie. Oznacza to, że jest ono bardzo istotne dla wszystkich. Zagłębiając się w jego znaczenie, nie sposób zapomnieć o ratownictwie medycznym i służbie zdrowia, dla których najważniejsze jest zdrowie i ludzkie życie. Gdy są one zagrożone, na pewno można zaobserwować pogotowie ratunkowe, jadące z włączonymi dźwiękami syren i sygnałem świetlnym na dachu pojazdu. 

\subsection{Pierwsza pomoc przedmedyczna w ramię ze zdrowiemi}
Pierwsza pomoc przedmedyczna jest istotna na kursach prawa jazdy i w szkole przed egzaminem na kartę rowerową i motorowerową. Wszystko jest po to, aby chronić ludzkie zdrowie, a zdrowie - to życie. 

\subsection{Zdrowie w indywidualnym podejściu}
Jest fundamentem ludzkiego szczęścia, chociaż w zależności od ludzkiego nastawienia do życia, z pewnością mogą pojawić się inne propozycje. W przypadku, gdy pojawiają się alternatywy na priorytet, to następnej pozycji nie można odmówić zdrowiu.

\section{Rozumienie zdrowia}
\subsection{Definicja zdrowia}
Po wstępnym rozważaniu na temat frazy "Zdrów jak ryba" i znaczeniem zdrowia w życiu, dobrym pomysłem jest pochylić się nad samym słowem zdrowie. Czym ono właściwie jest? Zdrowie jest przede wszystkim w momencie, gdy człowiek czuje się w takim stanie, jest usatysfakcjonowanym tym, że całokształt związany z nim jest w harmonii, nic mu nie dolega i nie boli. Jest to pojmowanie indywidualne, oprócz którego warto pamiętać o definiowaniu obiektywnym. Może być to brak choroby i niepełnosprawności.

\subsection{Podział zdrowia}
Zdrowie można wyszczególnić w podejściu obiektywnym na pewne grupy.

\subsubsection{Zdrowie fizyczne}
Zdrowie fizyczne jest pierwszym typem, cechującym się właściwym funkcjonowaniem całego ciała ze wszystkimi układami i narządami.

\subsubsection{Zdrowie psychiczne}
Drugim rodzajem jest zdrowie psychiczne, które można rozdrobnić na zdrowie emocjonalne, gdy osoba rozróżnia emocje, potrafi je okazać właściwie, ogarnia stres, napięcie, lęk, depresję, a nawet agresję. Poza tym logika i jasne myślenie określa zdrowie umysłowe.

\subsubsection{Zdrowie społeczne}
Człowiek jest stworzeniem stadnym, więc nawiązuje, utrzymuje i rozwija właściwie swoje więzi z innymi. To są określenia zdrowia społecznego.

\subsubsection{Zdrowie duchowe}
Ludzie, którzy wierzą i praktykują wyznawaną religię, a także ci z zasadami, zachowaniami i sposobami na spokój i równowagę wewnętrzną tworzą zbiór, tworzący zdrowie duchowe.

\subsection{Czynniki wpływające na zdrowie}
Na zdrowie wpływają z pewnością biologia, środowisko, styl życia i opieka zdrowotna. Siedzący tryb życia i niezdrowa dieta prowadzi między innymi do nadwagi i otyłości, a one są furtką do nadciśnienia tętniczego, cukrzycy i innych chorób, czyli utraty zdrowia. Zanieczyszczenia powietrza nie są też neutralne dla organizmu żyjącego. Trucizny podstępnie niszczą organizm, rozpoczynając od układu oddechowego, przez układ krążeniowy, kończąc niejednokrotnie na diagnozie nowotworu.

\section{Rola służby zdrowiai}
Opieka zdrowotna zmieniła się na przestrzeni lat. Wiele lekarstw zostało wycofanych z produkcji, wprowadzając nowe. Odkryto wiele pomocnych substancji, ratujących zdrowie, a nawet życie. Przykładem jest insulina, która jest hormonem, obniżającym poziom cukru we krwi. Osoby z cukrzycą umierały bez niego. W okresie II wojny światowej utrudniony dostęp do służb zdrowia, a także dzisiaj podczas wojny na Ukrainie można zaobserwować przykre konsekwencje jej niedoboru. Jest ona niezbędna i bezcenna dla zdrowia.

\section{Internetowe Konto Pacjenta}
\subsection{Początki IKP}
Służba zdrowia w Polsce, choć w wielu aspektach uległa poprawie, ma sfery nad którymi musi pracować. Okres pandemii koronawirusa zamknął ludzkość w domostwach, utrudniając dostęp do opieki zdrowotnej. Dobrym pomysłem, który ułatwiał realizacje recept, było Internetowe Konto Pacjenta, które ukazało się już przed COVID-19.

\subsection{Ścieżka do IKP}
IKP jest dostępem za pomocą profilu zaufanego, bankowości internetowej lub e-dowodu do historii recept, skierowań, wizyt lekarskich i certyfikatów.

\subsection{Zawartość Internetowego Konta Pacjenta}
\subsubsection{Recepty}
W receptach można z powodzeniem znaleźć recepty elektroniczne określane jako e-recepty z dawkowaniem leków, wykupionych recept od 2019 roku i leki wymagające recept, które zrefundował NFZ od 2008 roku.

\subsubsection{Skierowania}
Skierowania obejmują e-skierowania, czyli skierowania elektroniczne i informacje o skierowaniach do uzdrowisk w ramach NFZ.

\subsubsection{Historia leczenia}
W historii leczenia są zwolnienia lekarskie, także są te do zakładów pracy, co jest usprawnieniem i ułatwieniem dla pracowników i pracodawcy. Znajdują się też historie wizyt i informacje o endoprotezoplastykach stawowych w ramach NFZ od 2008 roku, a także lista zdarzeń medycznych od 2020 roku.

\subsubsection{Lekarstwa}
Ponadto jest lista leków, które zostały wykupione na e-receptę, lista wyrobów medycznych refundowanych, wyszukiwarka leków dostępnych w Polsce. Z powodzeniem można wyświetlić listę przyjętych dawek szczepionek z potencjalnym harmonogramem. 

\subsubsection{Uprawnienia}
Można też udostępnić swoje dane innym, nadać uprawnienia. Mogą to być osoby bliskie, zaufane, a także lekarze i pielęgniarki.

\subsubsection{Certyfikaty}
Każdy, kto przeszedł COVID-19 i nie tylko, także zaszczepieni mogą ujrzeć certyfikaty związane z COVID-19. W trakcie pandemii każdy, kto zachorował, był nosicielem wirusa, czy spędzał czas na kwarantannie lub izolacji domowej mógł wyświetlić decyzje odnośnie pobytu w domu, który często wiązał się z przymusowym odpoczynkiem od pracy, szkoły czy uczelni.

\subsubsection{Aplikacja mojeIKP}
Dostęp do wszystkich danych na IKP jest możliwy nie tylko z wyszukiwarki internetowej, lecz można z powodzeniem mieć wgląd w możliwości Internetowego Konta Pacjenta poprzez aplikacje mobilną na smartphone'a o nazwie mojeIKP. Smartphone z systemem Android oferuje dostęp do IKP po pobraniu oprogramowania ze sklepu Google Play, natomiast na platformie iOS jest dostępne mojeIKP z AppStore.\\

Aplikacja mojeIKP daje szereg możliwości oprócz dostępu z wyszukiwarki. Można otrzymać powiadomienia o nowych receptach i skierowaniach, wykupienie leków w aptece dzięki kodzie QR na ekranie bez podawania numeru PESEL, przeczytanie ulotki i dawkowania lekarstw, przypomnienia o zażywanych lekach, pobranie recept i skierowań w pliku PDF, sprawdzanie daty i lokalizacji wizyty gdy placówka umieściła dane w systemie, sprawne umówienie terminu wizyty, pobranie certyfikatów, dostęp do portalu Diety NFZ, rozwiązywanie codziennych quizów, by zweryfikować własną wiedzę o zdrowym żywieniu, krokomierz. Bez logowania można ponadto połączyć się z pogotowiem ratunkowym, odnaleźć numer alarmowy 112, numer Teleplatformy Pierwszego Kontaktu i numer Telefonicznej Informacji Pacjenta, publiczne powiadomienia, otrzymać przypomnienie o lekach po wczesnym ustawieniu grafiku z zalogowaniem.

\section{Potrzeba IKP}
Społeczeństwo przez wieki było różnorodne. Nigdy nie było teorii, która mówiłaby o tym, że świat składa się z klonów. Gdyby tak było, z pewnością świat byłby bardzo nudnym miejscem. Tak jest również z popytem na Internetowe Konto Pacjenta i aplikacją mojeIKP. Wszystko jest kwestią tego, jakie dana osoba odczuwa potrzeby. Nie ulega wątpliwości, że ułatwia to znacząco dbanie o własne zdrowie. Poza istotnymi informacjami można odnaleźć to, co ulepszy naszą dbałość o to, co mamy jedyne - zdrowie. Dotyczy to poza e-recept i e-skierowań również codziennego przyjmowania różnych specyfików. Ile ludzi jest na ziemi, tyle jest zróżnicowania w przyjmowaniu lekarstw. Jedynym czynnikiem, który wyznacza zapotrzebowanie na posiadanie omawianego konta jest ludzka potrzeba, by je mieć. Brak chęci prowadzi jednak do sporych ograniczeń włącznie z kontaktem telefonicznym pod numer przychodni, by uzyskać kod recepty do realizacji.
\end{document}